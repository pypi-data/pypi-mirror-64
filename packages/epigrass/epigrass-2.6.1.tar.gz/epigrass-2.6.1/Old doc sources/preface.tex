\chapter{Preface}
\lettrine{T}{he} emergence and re-emergence of infectious diseases and the evolution of resistant pathogens pose a great challenge for Public Health worldwide. Brazil, in particular, is facing increasing problems with the expansion, to urban centers, of diseases previously rural  (leishmaniosis), increasing mortality caused by previously benign diseases (dengue), potencial emergence of new diseases, and reduction of efficacy of traditional treatment protocols (tuberculosis, malaria, leprosy).

To deal with this situation, it is necessary to think strategically. How can we use our limited resources the best way possible? In other words, how to define public health policies for the use of chemicals and vaccines that optimize their impact in the short and long term? To develop such strategies, it is necessary to consider, and integrate, information from different sources, including biological information regarding host-parasite interaction, parasite response to chemicals or other control element, identification of risk situations for the population under study, identification of alternative control strategies, quantification of available resources, etc. The integration of all these data is often done in an incomplete way, resulting in sub-optimal decision making.

Important for the control of infectious diseases is the characterization of the temporal and spatial distribution of cases and risk factors. Climate, topography, human and vector population density are some features that may restrict diseases to certain geographical areas or seasons. In EpiGrass, the georefered space is the background for simulating intervention scenarios. Computer simulations are more and more used for evaluation of risk and formulation of disease control strategies \cite{EACasmanandBFischhoffandCPalmgrenandMJSmallandFWu2000}. These models are useful for determining, among other things, the expected number of cases in an epidemic (and the required medical cost), compare alternative control strategies (mass vaccination x localized vaccination, for example), for example. However, in general, simulation models do not take into consideration explicitly the spatial heterogeneity that characterizes epidemiological processes \ref{GregorySZaricandMargaretLBrandeau2002}. The integration of mathematical models to georefered data is strategic for increasing the applicability of mathematical models.

EpiGrass is a simulator which implements transmission models in a network where disease transmission occurs in the nodes (that may represent cities, neighborhoods or households) and spatial spread occurs via edges (that represent transportation routes or other forms contact networks). Examples of application of this approach can be found in \cite{BarrettCLandEubankSGandSmithJP2005,LaurenAncelMeyersandBabakPourbohloulandMEJNewmanandDanutaMSkowronskiandRobertCBrunham2005}.

In EpiGrass, epidemiological models take the form of network epidemiological models. Since decision in this complex context is often based on partial (and often poor) information, a measure of uncertainty is also necessary. This, associated with the stochastic nature of disease transmission makes stochastic models also an important tool.

EpiGrass  is not a part of the GRASS GIS system (\url{http://grass.itc.it}), but can make use of it as well as the R statistical package. GRASS GIS (Geographic Resources Analysis Support System) is a free software geographical information system , originally developed by the U.S. Army Construction Engineering Research Laboratories (USA-CERL, 1982-1995),  as a tool for land management and environmental planning.  Nowadays, GRASS has become a multi-national initiative, carried on by a team of developers from numerous locations in the world.  GRASS 5.7 (current release)  has more than  350 modules for management, processing, analysis and visualization of georeferenced data. GRASS GIS was chosen for four main reasons: 1) it provides free access to its internal code and algorithms; 2) availability of detailed documentation; 3) vast amount of code already available; 4) very active community of developers.  Statistical analysis in EpiGrass is done in R. The 'R data analysis programming language and environment' is an open source system for statistical computing and graphics (\url{http://www.r-project.org/}). R consists of a base package and a set of modules for  data handling and storage, calculations on arrays, several methods for data analysis, graphical facilities. It also has a well-developed  programming language which includes conditionals, loops, user-defined recursive functions and input and output facilities.

R and GRASS are integrated through a R/GRASS interface \cite{BivandRS2000}. Through this interface, R runs from within GRASS.  In the command line of GRASS, the user types R to open a R prompt. Within R, the R/GRASS interface is loaded by the commands:

\begin{lstlisting}[frame=trBL, caption=,label=]
> library(GRASS)
> grassobject <-gmeta()
> summary(grassobject)
\end{lstlisting}

The library GRASS contains a set of interface functions, including the definition of a data class named grassmeta, that can receive GRASS data.

EpiGrass is an opensource software. The financing for the EpiGrass comes from the Brazilian Research Council (CNPq), as part of a nation-wide initiative by the federal government url{http://www.iti.br/}, to develop and use free/opensource software as the standard computational platform throughout the country.

EpiGrass is the product of a colaboration between researchers from many disciplines: Fl�vio Coelho, Cl�udia Code�o and Oswaldo Gon�alves Cruz are epidemiological modellers at the Oswaldo Cruz Foundation (Rio de Janeiro - Brazil); Maria Goreti Rosa Freitas is an entomologist and epidemiologist working on spatial processes associated to Dengue Fever, at Oswaldo Cruz Foundation; Alexios Zavras is an expert in software development and an advocate of the Free Source Movement in Greece; Pantelis Tsouris is an engineer specialized in hardware development; Igor Cabral Correa is an enthusiastic and very smart programmer. We hope that, with this first release, this small community will grow and prosper.